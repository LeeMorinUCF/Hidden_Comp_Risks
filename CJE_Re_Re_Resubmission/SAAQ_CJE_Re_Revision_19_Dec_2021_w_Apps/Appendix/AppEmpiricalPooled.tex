\section{Empirical Results: Pooled Regressions}
\label{app:Pooled}


%We first conduct the regression using the full sample 
%(both male and female drivers).
%The results are shown in 
We first conduct the regression using the full sample
of both male and female drivers
shown in 
% 
Table \ref{tab:seas_Logit_vs_LPMx100K_pooled_regs}.
% 
The imposition of the policy increases the daily probability 
of receiving a ticket by $0.00387$ percentage points, 
which is very precisely estimated ($t$-statistic = $-94.054$). 
Once we add the age and policy interactions, 
the coefficient on the policy dummy is no longer significant, 
and heterogeneous effects by age group emerge. 
% 
The significance of the age and policy interactions
depends on the model specification. 
% 
Most of the effect is concentrated in the 16 to 19 and 20 to 24 age groups. The magnitude declines steadily for older age groups. 
The effect loses its statistical significance at the 0.1\% level once we reach the 55 to 64 age group.
% 
The average marginal effects are slightly lower than the coefficients 
from the linear probability model, although qualitatively similar. 
In contrast, the marginal effects are three to four times as high
for representative drivers who tend to get tickets. 

% Logistic Regression and Linear Probability Models: Seasonal Logit and LPM x 100K Specification for All Drivers by Point Value 

\begin{table}% [ht] 
\centering 
\begin{tabular}{l r r r r l r r l} 

\hline 
 
 & \multicolumn{5}{c}{Logistic Regression}  & \multicolumn{3}{c}{Linear Probability Model} \\ 

 \cmidrule(lr){2-6}\cmidrule(lr){7-9} 
 & \multicolumn{2}{c}{Marginal Effects} & Estimate & Standard & Sig. & Estimate & Standard & Sig. \\ 

 \cmidrule(lr){2-3} 
 &   AME &  MER  &          &  Error   &      &          &  Error   &     \\ 

\hline 
 
\multicolumn{8}{l}{\textbf{Full sample Drivers} (9,675,245,494 observations)} \\ 

\hline
\multicolumn{8}{l}{Model without age-policy interaction: } \\ 
Policy                   &  -3.7849        &  -17.8748       &  -0.0926        &  0.0010       &   **       &  -3.8656        &  0.0411       &   **       \\ 
\hline
\multicolumn{8}{l}{Model with age-policy interaction: } \\ 
Policy                   &  -0.6709        &  -2.9587       &  -0.0435        &  0.0370       &            &  -1.1761        &  0.5700       &            \\ 
Age 16-19 * policy   &  -4.9094        &  -13.7384       &  -0.0684        &  0.0373       &            &  -6.2697        &  0.6707       &   **       \\ 
Age 20-24 * policy   &  -5.3116        &  -14.2921       &  -0.0822        &  0.0371       &            &  -6.7723        &  0.6059       &   **       \\ 
Age 25-34 * policy   &  -3.9081        &  -12.1981       &  -0.0834        &  0.0370       &            &  -5.1489        &  0.5805       &   **       \\ 
Age 35-44 * policy   &  -1.7999        &  -6.0114       &  -0.0430        &  0.0371       &            &  -2.6807        &  0.5780       &   **       \\ 
Age 45-54 * policy   &  -1.1679        &  -4.2241       &  -0.0337        &  0.0371       &            &  -1.9497        &  0.5759       &    *       \\ 
Age 55-64 * policy   &  -0.6156        &  -2.4087       &  -0.0225        &  0.0371       &            &  -1.2160        &  0.5762       &            \\ 
Age 65+ * policy   &  0.7289        &  3.1682       &  0.0385        &  0.0372       &            &  0.3767        &  0.5752       &            \\ 

\hline 

\end{tabular} 
\caption{Pooled Regressions for all offences, male and female drivers} 
For each regression, the dependent variable is an indicator that a driver has committed  
any offence on a particular day.  
All regressions contain age category and demerit point category controls, 
as well as monthly and weekday indicator variables. 
The baseline age category comprises drivers under the age of 16. 
The heading ``Sig.'' is an abbreviation for statistical significance, with 
the symbol * denoting statistical significance at the 0.1\% level 
and ** the 0.001\% level. 
In the linear probability model, coefficients and heteroskedasticity-robust standard errors are  
multiplied by 100,000.  
\label{tab:seas_Logit_vs_LPMx100K_pooled_regs} 
\end{table} 
 



Although running the pooled regressions in 
Table \ref{tab:seas_Logit_vs_LPMx100K_pooled_regs}
is inconsistent with our theoretical model and our preliminary data analysis, 
we present these results to highlight three issues. 
% 
First, the pooled regressions measure the overall policy effect, 
which can be compared to other policy analyses that do not
consider gender differences. 
% 
Second, the misspecification induced by pooling serves as a robustness check for the interactions by age. Indeed, in the logistic model, the age-policy effect is specified as a proportional change in probability and therefore not precisely estimated. However, in the linear probability model, when it is specified as a constant effect independent of the other covariates, it is precisely estimated.
%
Finally, while the proportional changes measured by the logistic model already account for differences by age group, these differences can only be accommodated by introducing interaction terms in the linear model.

% Monster apologetic paragraph removed and replaced:
%
%Finally, a nonlinear model measures 
%differently the second-order interaction effects, such as the age-policy effect.
%% 
%The coefficients in the interaction
%include terms such as the second cross-partial derivative 
%of the probability, with respect to the pair of covariates. 
%%
%\citet{ainorton2003} demonstrate this and caution that 
% the interaction effect in a logistic model 
%is not correctly characterized by the
%sign, magnitude, or statistical significance of the coefficient on the
%interaction term;
%the coefficients in the linear probability model measure the marginal effects
%without confounding with the extra terms. 
%% 
%In summary, these coefficients measure different quantities
%and one would not expect them to have the same values 
%or pattern of significance. 
%% 
%Despite these differences, the marginal effects from both models 
%in Table \ref{tab:seas_Logit_vs_LPMx100K_pooled_regs}
%%
%are consistent with each other in that every marginal effect 
%from the logistic model is within two standard errors
%of the estimates from the linear probability model.
%% 
%One explanation for the differences between the models 
%is that the age differences in policy effect are not as robust
%as the differences in policy effects by gender, which we explore next. 


% Logistic Regression and Linear Probability Models: Seasonal Logit and LPM x 100K Specification for All Drivers by Age Group 

\begin{table}% [ht] 
\centering 
\begin{tabular}{l r r r r l r r l} 

\hline 
 
 & \multicolumn{5}{c}{Logistic Regression}  & \multicolumn{3}{c}{Linear Probability Model} \\ 

 \cmidrule(lr){2-6}\cmidrule(lr){7-9} 
 & \multicolumn{2}{c}{Marginal Effects} & Estimate & Standard & Sig. & Estimate & Standard & Sig. \\ 

 \cmidrule(lr){2-3} 
 &   AME &  MER  &          &  Error   &      &          &  Error   &     \\ 

\hline 
 
\multicolumn{8}{l}{\textbf{Drivers in Age Group 16-19} (252,903,500 observations)} \\ 

Policy Indicator        &  -6.6974        &  -13.0280       &  -0.0853        &  0.0045       &   **       &  -6.5742        &  0.3539       &   **       \\ 

\hline 

\multicolumn{8}{l}{\textbf{Drivers in Age Group 20-24} (679,154,662 observations)} \\ 

Policy Indicator        &  -8.5951        &  -23.2737       &  -0.1199        &  0.0029       &   **       &  -8.4513        &  0.2059       &   **       \\ 

\hline 

\multicolumn{8}{l}{\textbf{Drivers in Age Group 25-34} (1,721,109,220 observations)} \\ 

Policy Indicator        &  -6.6050        &  -20.6684       &  -0.1268        &  0.0021       &   **       &  -6.5547        &  0.1102       &   **       \\ 

\hline 

\multicolumn{8}{l}{\textbf{Drivers in Age Group 35-44} (1,957,261,955 observations)} \\ 

Policy Indicator        &  -3.9187        &  -12.7882       &  -0.0877        &  0.0021       &   **       &  -3.9221        &  0.0956       &   **       \\ 

\hline 

\multicolumn{8}{l}{\textbf{Drivers in Age Group 45-54} (2,171,413,198 observations)} \\ 

Policy Indicator        &  -3.0705        &  -11.3413       &  -0.0837        &  0.0022       &   **       &  -3.0670        &  0.0822       &   **       \\ 

\hline 

\multicolumn{8}{l}{\textbf{Drivers in Age Group 55-64} (1,611,824,607 observations)} \\ 

Policy Indicator        &  -2.2195        &  -9.7673       &  -0.0775        &  0.0030       &   **       &  -2.2167        &  0.0843       &   **       \\ 

\hline 

\multicolumn{8}{l}{\textbf{Drivers in Age Group 65-199} (1,262,602,202 observations)} \\ 

Policy Indicator        &  -0.4329        &  -2.5930       &  -0.0232        &  0.0041       &   **       &  -0.4337        &  0.0768       &   **       \\ 

\hline 

\end{tabular} 
\caption{Regressions for all offences, by age group} 
For each regression, the dependent variable is an indicator that a driver has committed  
any offence on a particular day.  
All regressions contain age category and demerit point category controls, 
as well as monthly and weekday indicator variables. 
The baseline age category comprises drivers under the age of 16. 
The heading ``Sig.'' is an abbreviation for statistical significance, with 
the symbol * denoting statistical significance at the 0.1\% level 
and ** the 0.001\% level. 
In the linear probability model, coefficients and heteroskedasticity-robust standard errors are  
multiplied by 100,000.  
\label{tab:seas_Logit_vs_LPMx100K_regs_by_age} 
\end{table} 
 


This key difference between the two models is worth explaining in detail, 
as it also clarifies the calculation of the marginal effects. 
%In the linear probability model, the coefficient on the age-policy effect 
%represents the second cross-partial derivative of the predicted probability 
%with respect to the age indicator and the policy indicator. 
%
% In the logistic regression model, the same derivative is split into several pieces. 
For brevity, let $A_{it}$ denote the indicator for a particular age group 
among the categorical variables $\bm{agecat}_{it}$
and let $\bm{\beta}_j$ and $\bm{X}$ denote 
the remaining coefficients and explanatory variables 
with $\bm{X}_{it} = [\bm{1}, \bm{ptsgrp}_{it}, \bm{calendar}_{it}]$. 
%The change in probability for driver $i$ on day $t$ is found by taking
%discrete differences:
% 
The marginal effects, AME and MER, 
were calculated as the treatment effect following \citet{puhani2012}:
the cross difference of the observed outcome 
minus the cross difference of the potential non-treatment outcome. It corresponds to the incremental effect of the interaction term coefficients. 
In our notation, with $j$ subscripts suppressed for the pooled regression, 
this treatment effect, in the AME and MER, equals
$$
	F(\beta_D + \beta_A + \beta_{D\cdot A} + \bm{\beta}^\prime \bm{X}_{it})
		- F(\beta_D + \beta_A + \bm{\beta}^\prime \bm{X}_{it}).
$$
This expression is different from the cross difference of the observed outcome in the nonlinear model: 
$$
	F(\beta_D + \beta_A + \beta_{D\cdot A} + \bm{\beta}^\prime \bm{X}_{it})
		- F(\beta_D + \bm{\beta}^\prime \bm{X}_{it})
		- F(\beta_A + \bm{\beta}^\prime \bm{X}_{it})
		+ F(\bm{\beta}^\prime \bm{X}_{it}).
$$
To obtain the treatment effect, we subtract 
the cross difference of the potential non-treatment outcome:
$$
	F(\beta_D + \beta_A + \bm{\beta}^\prime \bm{X}_{it})
		- F(\beta_D + \bm{\beta}^\prime \bm{X}_{it})
		- F(\beta_A + \bm{\beta}^\prime \bm{X}_{it})
		+ F(\bm{\beta}^\prime \bm{X}_{it}), 
$$
which is the cross difference in the nonlinear model 
without an interaction term; it is nonzero in general, 
but is zero in the linear model. 
% 
This point is raised by  
\citet{ainorton2003}, who caution that 
the interaction effect in a logistic model 
is not correctly characterized by the
sign, magnitude, or statistical significance of the coefficient on the
interaction term.
%;
%the coefficients in the linear probability model measure the marginal effects
%without confounding with the extra terms. 
Intuitively, in the logistic regression model, the policy effect $\beta_D$
already has a different proportional effect on the predicted probability, 
proportional to the $\bm{\beta}_A$ coefficients, 
even when the interaction term $\beta_{D\cdot A}$ is zero. 
For drivers in the respective age groups,
the coefficients in $\bm{\beta}_{D\cdot A}$ measure the effect of the policy
\emph{in excess of} the proportional effect of $\beta_D$
in the logistic regression model. 
In sum, there is no reason to believe the coefficients $\bm{\beta}_{D\cdot A}$
should match in statistical significance between the models. 

To provide further evidence of the concordance between the models, 
we conduct separate regressions by age group. 
These estimates are presented in 
Table \ref{tab:seas_Logit_vs_LPMx100K_regs_by_age}. 
The overall policy effect of approximately -3.8 per 100,000 driver days (see table \ref{tab:seas_Logit_vs_LPMx100K_pooled_regs}) 
closely matches the coefficients for the drivers aged 35 to 44. 
In absolute terms, the coefficients are highest for the 20 to 24 age group and then decrease for older age groups, thus
matching the pattern in Table \ref{tab:seas_Logit_vs_LPMx100K_pooled_regs}. 
These results are consistent 
with our theoretical model in Section \ref{sec:Model}, 
since younger drivers engage in more risky driving behaviour. 
