

In 2008, the province of Quebec drastically increased penalties for speeding 
well above the speed limit by doubling fines and instituting on-the-spot licence suspension. 
Using administrative driving and licensing records in Quebec from 2006 to 2010, 
we examine whether the new law discouraged unlawful driving behaviour 
by investigating the frequency with which motorists received traffic citations. 
We find that the new law was effective in deterring motorists from speeding.
Moreover, the effect was most pronounced for males compared to females, 
for young compared to old, 
and especially so for drivers with high demerit point balances accumulated from past
infractions compared to those with few or no tickets. 
In sum, the change in behaviour was most apparent for 
those drivers who were the intended targets for the legislation. 
%We also find evidence for the change in behaviour at the intensive and extensive margins:
%drivers got fewer tickets overall
%but relatively fewer tickets for lesser offences, 
%after the change in penalties. 

% We find that males became less likely to get traffic tickets after the law came into effect
% with the largest effects on those aged 16 to 24. 
% However, females largely did not change their behaviour in response to this new law. 
% The effect on the behaviour of female drivers, in contrast, was much smaller than that for males.

\medskip
\noindent
Keywords: driving behaviour, law enforcement, risk aversion, speeding.
