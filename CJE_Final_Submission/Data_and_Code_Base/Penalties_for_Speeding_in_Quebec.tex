\documentclass[11pt]{paper}
\usepackage{fullpage}
\usepackage{hyperref}

\begin{document}


\phantom{0}
\vspace{1.0in}


\begin{centering}

{\huge \it
Data Availability Guidelines and Code Base  \\
\bigskip
for \\
\bigskip
``Penalties for Speeding and their Effect on Moving Violations: 
Evidence from Quebec Drivers'' \\
}

\vspace{1.25in}


{\large 
Vincent Chandler \\
{\it Universit\'{e} du Qu\'{e}bec en Outaouais} \\
\medskip
Lealand Morin \\
{\it University of Central Florida} \\
\medskip
Jeffrey Penney \\
{\it University of Alberta } \\
}

\vspace{1.25in}



\today

\end{centering}


\pagebreak

\section*{Penalties\_for\_Speeding\_in\_Quebec}

This is the documentation for the code base to accompany the manuscript 
{\it Penalties for Speeding and their Effect on Moving Violations: 
Evidence from Quebec Drivers} 
by Chandler, Morin, and Penney in the \emph{Canadian Journal of Economics}, 2022

Any updates will be available on the GitHub code repository 
 \texttt{Penalties\_for\_Speeding\_in\_Quebec}
available at the following link: \\

 \texttt{https://github.com/LeeMorinUCF/Penalties\_for\_Speeding\_in\_Quebec} \\


\section*{Data Availability}

All data were obtained from the 
Soci\'{e}t\'{e} de l'assurance automobile du Qu\'{e}bec (SAAQ), 
the driver's licence and insurance agency 
for the province of Qu\'{e}bec. 
The administrators responsible for the data can be reached at
\url{https://saaq.gouv.qc.ca/en/reach-us}
or at the following mailing address: \\
 
\noindent
Service de la recherche en s\'{e}curit\'{e} routi\`{e}re \\
Soci\'{e}t\'{e} de l'assurance automobile du Qu\'{e}bec \\
333, boulevard Jean-Lesage, C-4-12 \\
C. P. 19600, succursale Terminus \\
Qu\'{e}bec (Qu\'{e}bec) G1K 8J6 \\
T\'{e}l\'{e}phone : 418 528-4095 \\

Once the data are obtained, 
these primary datasets must be placed in the \texttt{Data}
folder before running the scripts.


\subsection*{Traffic Tickets}

The primary data source is an anonymized record of traffic tickets from the SAAQ for each year in the sample. 
The data were provided to the authors under an understanding
that the data not be made publicly available. 

The official name of the database is 
\emph{Fichiers des infractions au Code de la s\'{e}curit\'{e} routi\`{e}re}. 
The datasets are named in the format \texttt{csYYYY.dta}, 
with \texttt{YYYY} indicating the year in which drivers received tickets. 
The datasets contain the following variables.

\begin{itemize}

\item \texttt{pddobt} 
or \emph{Nombre de points} 
is the number of demerit points awarded for the offence.
\item \texttt{dinf} 
or \emph{Date d'infraction}
is the date of infraction in \texttt{YYYY-MM-DD} format.
\item \texttt{dcon}
or \emph{Date de condamnation} 
is the date of conviction in \texttt{YYYY-MM-DD} format.
\item \texttt{seq} 
or \emph{Num\'{e}ro s\'{e}quentiel} 
is a sequence of unique identification numbers for the drivers.

\end{itemize}


\subsection*{Statistics for Individual Drivers}

The above record of tickets are marked with
driver-specific identifier, which serves as a key for 
a dataset of driver-specific characteristics.
This dataset contains the driver identification number, 
along with the gender and date of birth of each driver.
This information is not publicly available
to protect the privacy of the drivers.

The official name of the database is 
\emph{Fichier des num\'{e}ros s\'{e}quentiels}. 
The file \texttt{seq.dta} contains licensee data 
for 3,911,743 individuals who received tickets
and includes the following variables.

\begin{itemize}

\item \texttt{seq} 
or \emph{Num\'{e}ro s\'{e}quentiel} 
is a sequence of unique identification numbers for the drivers.
\item \texttt{sxz} 
or \emph{Sexe}
is either 1.0 or 2.0, an indicator for male or female, respectively.
\item \texttt{an} 
or \emph{Ann\'{e}e de naissance provenant du num\'{e}ro de permis de conduire}
is an integer for the year of birth of each driver.
\item \texttt{mois} 
or \emph{Mois de naissance provenant du num\'{e}ro de permis de conduire}
is an integer for the month of birth of each driver.
\item \texttt{jour} 
or \emph{Jour de naissance provenant du num\'{e}ro de permis de conduire}
is an integer for the calendar day of birth of each driver.

\end{itemize}


\subsection*{Aggregate Counts of Drivers}


Counts of individual drivers were obtained 
from the Website of the 
Banque de donn\'{e}es des statistiques officielles sur le Qu\'{e}bec, 
available at
\url{https://bdso.gouv.qc.ca/pls/ken/ken213_afich_tabl.page_tabl?p_iden_tran=REPERRUNYAW46-44034787356%7C@%7Dzb&p_lang=2&p_m_o=SAAQ&p_id_ss_domn=718&p_id_raprt=3370#tri_pivot_1=500400000}. 


The statistics were compiled into a single spreadsheet
\texttt{SAAQ\_drivers\_annual.csv}, 
which is available in the \texttt{Data} folder
and contains the following variables. 


\begin{itemize}

\item \texttt{age\_group} is an age range in years.
\item \texttt{sex} is an indicator for the gender of drivers, 
  either \texttt{"M"} or \texttt{"F"}.
\item \texttt{yrYYYY} denotes that the column records the number of drivers in each year \texttt{YYYY} on June 1 of each year.

\end{itemize}


\section*{Instructions}

All regression results, tables and figures in the manuscript
can be obtained by running the shell script \texttt{SAAQ\_CJE.sh}. 
The workflow proceeds in three stages: 
one set of instructions outlines the operations 
to transform the raw data in the 
SAAQ database into the dataset that is the input 
for the statistical analysis
in the next stage. 
In the final stage, the estimation results 
are used to create the figures and tables for the manuscript. 


\section*{Data Preparation}

Run the scripts in the \texttt{Code/Prep} folder, which 
perform the following operations:


\begin{enumerate}

\item Run the R script \texttt{SAAQ\_tickets.R}, which
	collects the record of tickets for each year into a 
	single dataset of tickets. 
	 This produces the dataset \texttt{SAAQ\_tickets.csv}, 
	  which is the record of events in the regression models. 

\item Run the R script \texttt{SAAQ\_point\_balances.R}, which
  calculates the accumulated demerit point balances
  for each driver and collects counts of drivers at each 
  demerit point level. 
	 This produces the dataset \texttt{SAAQ\_point\_balances.csv}, 
	  which is the record of counts of drivers at each 
  demerit point level for each day in the sample period.
  This is the record of non-events for the subset of drivers
  \emph{who have ever received tickets}. 

\item Run the R script \texttt{SAAQ\_driver\_counts.R}, which
  collects the public record of the number of drivers in 
  each gender and age group category. 
  It uses linear interpolation to transform 
  the dataset \texttt{SAAQ\_drivers\_annual.csv}
  into a record of daily counts \texttt{SAAQ\_drivers\_daily.csv}. 
	 This dataset is the the record of non-events for the subset 
	 of drivers   \emph{who have never received tickets}. 


\item Run the R script \texttt{SAAQ\_join.R}, which
  joins the above datasets into the complete record of 
  events and non-events for all drivers in Quebec. 
	 This produces the dataset \texttt{SAAQ\_full.csv}, 
	  which is used in the regression analysis in the next stage.
  
\end{enumerate}


\section*{Statistical Analysis}

The script in the \texttt{Code/Reg} folder  
is the main script for 
the sequence of regression models. 

Run this script, \texttt{SAAQ\_Regs.R}, which
estimates all models in the paper in a series of loops. 
It perform the following operations:



\begin{enumerate}

\item  Read in the main dataset \texttt{SAAQ\_full.csv}.
\item  Create and modify categorical variables.
\item  Define the policy indicator
    to represent the change in legislation on April 1, 2008
    and the sample period 
    over the four-year period centered on this date. 
\item  Defines the sequence of sets of models to be estimated, 
    including the full sample, high-point drivers, 
    an event study and an analysis by demerit point balances,
    as well as placebo regressions. 

\end{enumerate}


For each model, the script performs the following operations: 

\begin{enumerate}

\item  Define the target variable. 
\item  Set the relevant sample period, 
    which differs for the placebo regression.
\item  Set the sample selection, 
    to select male or female drivers
    and to select either the full sample or high-point drivers.
\item  Estimate the linear and logistic regression model. 
\item  Calculate HCCME standard errors 
    for the linear probability model. 
\item  Calculate the marginal effects for the relevant coefficients. 
\item  Save the estimation results in files stored in the
    \texttt{Estn} folder to produce tables and figures 
    for the manuscript. 

\end{enumerate}



\section*{Manuscript}


Once the estimates are obtained, 
run a series of scripts 
to draw from values in the estimation
results to produce the figures and tables in the manuscript. 

\subsection*{Producing the Output}

Run the scripts in the \texttt{Code/Out} folder
perform the following operations:


\begin{enumerate}

\item  Run the script \texttt{SAAQ\_Tables.R}, which
    produces tables of estimates from the results in
    the \texttt{Estn} folder. 
    These tables are all output to the \texttt{Tables} folder. 
\item  Run the script \texttt{SAAQ\_Estn\_Figs.R}, which
    produces the
    figures from the estimation of the event studies and 
    the estimation with granular demerit-point categories.
    These figures are output to the \texttt{Figures} folder
    and are ultimately named 
    \texttt{Figure3.eps} and \texttt{Figure4.eps}. 
\item  Run the script \texttt{SAAQ\_Count\_Figs.R}, which 
    produces the
    figures of the frequency of tickets
    from aggregate data by month. 
    This produces \texttt{num\_pts\_5\_10.eps} 
    and \texttt{num\_pts\_7\_14.eps},
    which are both output to the \texttt{Figures} folder
    and are ultimately named 
    \texttt{Figure1.eps} and \texttt{Figure2.eps}. 
    It also outputs a dataset 
	\texttt{Point\_Freq\_Gender\_Ratio.csv}, 
	which is used to calculate
    the summary statistics in Table 2. 

\end{enumerate}




\subsection*{Producing the Tables Separately}

All tables in the manuscript were output to the folder \texttt{Tables}.

\begin{enumerate}

\item  Table 1 was produced manually 
    and appears in the file \texttt{Penalties.tex}.
\item  Table 2 was produced by an Excel spreadsheet
    \texttt{Point\_Freq\_Gender\_Ratio.xlsx} from the outputs in 
    \texttt{Point\_Freq\_Gender\_Ratio.csv} and appears in the file 
    \texttt{Point\_Freq\_Gender\_Ratio.tex}.
\item  Tables 3, 4, 5, 6 and 7 were produced together from
    the commands on lines 248 to 258 of the script \texttt{SAAQ\_Tables}
    using the regression results obtained above
    and the function library \texttt{SAAQ\_Tab\_Lib.R} 
    in the folder \texttt{Code/Lib}. 

\end{enumerate}





\subsection*{Producing the Figures Separately}

All figures in the manuscript were output to the folder \texttt{Figures}.

\begin{enumerate}

\item  Figure 1 was produced from
    the commands on lines 263 to 284 
    of the script \texttt{SAAQ\_Count\_Figs.R}.
\item  Figure 2 was produced from
    the commands on lines 310 to 331 
    of the script \texttt{SAAQ\_Count\_Figs.R}.
\item  Figure 3 was produced from
    the commands on lines 156 to 189 
    of the script \texttt{SAAQ\_Estn\_Figs.R}
    using the regression results obtained above.
\item  Figure 4 was produced from
    the commands on lines 252 to 292 
    of the script \texttt{SAAQ\_Estn\_Figs.R}
    using the regression results obtained above.

\end{enumerate}



\section*{Libraries}


The above programs use functions defined in the following libraries, which are stored in the \texttt{Code/Lib} folder. 


\begin{itemize}

\item  The script \texttt{SAAQ\_Agg\_Reg\_Lib.R} defines functions
    for running regressions with data aggregated by the 
    number of driver days for each combination of the 
    dependent variables. 
    Since weighted regression is used in different contexts, 
    this library makes adjustments, 
    such as for degrees of freedom,
    to make the results equivalent to those which would be obtained
    from the full dataset with one observation per driver per day. 
    Since most drivers do not get tickets on most days,
    this library effectively compresses the dataset
    by a factor of one thousand, 
    from billions of driver days to millions of unique observations.
\item  The script \texttt{SAAQ\_Agg\_Het\_Lib.R} 
    defines functions for the 
    calculation of heteroskedasticity-corrected standard errors
    with aggregated data.
\item  The script \texttt{SAAQ\_Reg\_Lib.R} defines helper functions
    for data formatting and preparation for regressions. 
\item  The script \texttt{SAAQ\_MFX\_Lib.R} defines functions
    to calculate marginal effects. 
\item  The script \texttt{SAAQ\_Tab\_Lib.R} defines functions
    to generate \LaTeX tables from regression results. 


\end{itemize}



\section*{Computing Requirements}

All the tables
and figures in the paper can be performed on a single microcomputer, 
such as a laptop computer.
The particular model of computer 
on which the statistical analysis was run
is a 
Dell Precision 3520,
running a 64-bit Windows 10 operating system, 
with a 4-core x64-based processor,
model Intel(R) Core(TM) i7-7820HQ CPU, 
running at 2.90GHz, 
with 16 GB of RAM.


\section*{Software}


The statistical analysis was conducted in R, version 4.0.2,
which was released on June 22, 2020, 
on a 64-bit Windows platform x86\_64-w64-mingw32/x64. 

The attached packages include the following:

\begin{itemize}

\item \texttt{foreign} version 0.8-81, to open datasets in \texttt{.dta} format. 

\item \texttt{data.table}, version 1.13.0 (using 4 threads), 
to handle the main data table for 
data preparation and analysis 
in the scripts in the \texttt{Code/Prep} and \texttt{Code/Reg} folders. 

\item \texttt{xtable}, version 1.8-4, to generate \LaTeX tables for 
Tables 3, 4, 5, 6, and 7.   

\item \texttt{scales} version 1.1.1, to format numbers in \LaTeX tables.


\end{itemize}



Upon attachment of the above packages, 
the following packages were loaded via a namespace, but not attached,
with the following versions:

\begin{itemize}

\item \texttt{Rcpp} version 1.0.5
\item \texttt{RcppParallel} version 5.0.2
\item \texttt{parallel} version 4.0.2
\item \texttt{compiler} version 4.0.2
\item \texttt{pkgconfig} version 2.0.3
\item \texttt{haven} version 2.3.1
\item \texttt{stringr} version 1.4.0
\item \texttt{withr} version 2.4.2    
\item \texttt{tidyr} version 1.1.3       
\item \texttt{car} version 3.0-10        
\item \texttt{scales} version 1.1.1         
\item \texttt{stringi} version 1.5.3    

\end{itemize}


\section*{Acknowledgements}


The authors would like to thank 
Fran\c{c}ois Tardif for his help with the data in the early stages of this project, 
as well as  
Catherine Maclean 
for helpful suggestions and valuable comments. 
Jeffrey Penney acknowledges support from SSHRC. 
The authors are especially grateful to the editor and two anonymous referees 
for comments and suggestions that led to substantial improvements from the original manuscript. 
The authors have no conflict of interest to disclose. 
The usual caveat applies.



\section*{References}

\noindent Soci\'{e}t\'{e} de l'assurance automobile du Qu\'{e}bec,
\noindent{\it Fichiers des infractions au Code de la s\'{e}curit\'{e} routi\`{e}re}, 
1998--2010, 
accessed February 2012. \\

\noindent Soci\'{e}t\'{e} de l'assurance automobile du Qu\'{e}bec,
\noindent{\it Fichier des num\'{e}ros s\'{e}quentiels}, 
1998--2010, 
accessed February 2012. \\


\noindent Banque de donn\'{e}es des statistiques officielles sur le Qu\'{e}bec, 
\emph{Nombre de titulaires d'un permis de conduire ou d'un permis probatoire selon le sexe et l'\^{a}ge, Qu\'{e}bec et r\'{e}gions administratives}
\url{https://bdso.gouv.qc.ca/pls/ken/ken213_afich_tabl.page_tabl?p_iden_tran=REPERRUNYAW46-44034787356%7C@%7Dzb&p_lang=2&p_m_o=SAAQ&p_id_ss_domn=718&p_id_raprt=3370#tri_pivot_1=500400000}. 






\end{document}
