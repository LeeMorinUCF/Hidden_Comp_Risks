\section{Introduction}
\label{sec:Introduction}

In 2018, 1,743 individuals died in Canada following a car accident 
\citep{transcan2018}. 
Such accidents are the leading cause of death for individuals aged between 15 and 44
\citep{statscan2020}.   
Many OECD countries have recently introduced harsher punishments to deter 
the types of behaviour that increase the likelihood of such tragedies. 
For example, penalties for speeding well above the speed limit now typically include 
some combination of substantially increased fines, immediate vehicle seizure, and licence suspension.
These laws are typically referred to as excessive speeding laws or stunt-driving laws. 

Quebec followed this trend and introduced excessive speeding penalties in 2008. 
Its provisions are triggered when driving well above the speed limit. 
For example, driving at a speed of 100km/h in a 60km/h zone would be considered excessive speeding.
These harsher punishments received widespread media coverage 
both before and after the change in legislation, 
and there was a sustained campaign by the provincial government 
to help ensure drivers were aware of the law. 
The Quebec government has since declared the legislation to be successful, 
with the number of excessive speeding tickets decreasing over time.%
\footnote{The legislation has been unsuccessfully challenged in court.}
%
Furthermore, the number of accidents with bodily harm 
decreased from 36,816 in 2006 to 32,371 in 2010, 
and the number of accidents with fatalities decreased from 666 to 441 
over the same period 
\citep{saaq2011}. 
Even though these findings are encouraging, 
they don’t provide conclusive evidence that the harsher penalties truly caused changes in behaviour. 


Since 
\citet{becker1968b}, 
economists have theorized that harsher punishments 
alter the incentives of individuals and thus ultimately their behaviour. 
\citet{hellandtabarrok2007} 
provide evidence for this mechanism 
studying the impact of California’s three-strike legislation on the recidivism rates of felons. 
However, the effectiveness of deterrence is unclear in the context of driving. 
Indeed, 
\citet{bourgeonpicard2007} 
hypothesize that some drivers may be impossible to deter 
either because they do not care about the penalties or because they are not aware they are speeding.


A broad literature has investigated the role of deterrence on driving 
and particularly alcohol consumption 
\citep[e.g.][]{hansen2015}, 
but less attention has been devoted to speeding. 
Some empirical research has determined the impact of very influential policies 
like the introduction of a demerit point system. 
For example, 
\citet{bennedittiniNicita2009} 
show a reduction in road fatalities 
through deterrence and incapacitation following the introduction of such a system. 
More generally, 
\citet{castillocastro2012} 
provide a meta-study 
demonstrating the broad positive impact of such a policy in a variety of countries. 
In Quebec, 
\citet{dionneetal2011} 
focus on the threat of the loss of licence 
on the behaviour of a driver close to the demerit point threshold of suspension. 
They find a reduction in the probability of violation for drivers 
with a large number of demerit points and conclude 
the system is successful in deterring the worst offenders. 
Finally, closest to this paper, 
\citet{meirambayeva2014} 
show the introduction of a street-racing law in Ontario decreased the number of accidents 
by conducting an intervention analysis with an ARIMA model of 
the monthly number of accidents in Ontario.
%
Using monthly, aggregated data, 
they find an intervention effect for young males
but not for females or mature males. 


In this paper, we investigate the effect of Quebec’s excessive speeding legislation
on the frequency and types of violations incurred by Quebec drivers 
using an event-study design. 
Such violations are a proxy for driving behaviour, 
so this study will glean insight into the effect of increasing penalties on dangerous driving. 
It is important to note that we are looking at all violations which result in demerit points, 
and not just those that are affected by the change in the law. 
% 
In contrast to \citet{meirambayeva2014}, 
our focus on traffic violations
studies an event further up the chain of causation that happens more frequently. 
Although these events are still rare,
we can measure gender and age differences more precisely, 
using a large dataset of individual drivers at the daily frequency. 

We analyze driving records obtained from administrative data sets 
of the Government of Quebec comprising the universe of violations 
from 2006 to 2010 and records on drivers’ licences over the same period. 
The use of large administrative datasets is necessary because only a small fraction 
of all drivers are impacted by the policy change; 
yet, these drivers are particularly important because they are 
generally responsible for accidents causing bodily harm and property damage. 
% 
We then present a simple theoretical model to examine the predictions of economic theory 
on the effect of the law on drivers.
%
The model predicts the possibility of heterogeneous effects by age and gender,  
which guides our empirical specification to test these predictions. 
% 
We examine the heterogeneous effects of the excessive speeding law 
on both the extensive margin (getting a ticket) 
and the intensive margin (getting a more severe ticket) across gender and age. 


We find that the daily probability of receiving a ticket (extensive margin) 
decreases after the implementation of the law. 
When we examine the results by age group, we find that the effects vary substantially, 
with young drivers between the ages of 16 and 24 being the most affected by the law, 
while there is little effect for drivers over the age of 45. 
Repeating the analysis by gender, we see that both males and females change their behaviour, 
but the magnitude of the effect on males is about eight times that of females. 
Examining the breakdown by age categories, we see that the effect gradually declines 
for males until age 55, while there appears to be no age effect for females. 

We then investigate the effect on the intensive margin. 
For males, the probability of getting tickets worth only one demerit point 
actually increases following the new policy, while tickets for all other point values decrease. 
This result suggests that male drivers are still exceeding the speed limit 
but are driving more slowly than before the introduction of the legislation. 
A similar pattern exists for female drivers for one and two point violations. 
%
We conclude that Quebec’s 2008 excessive speeding law has had substantial spillover effects 
on both the extensive and intensive margins of driving behaviour. 
In other words, not only has it reduced the number of drivers driving well above the speed limit, 
it has also led to a decrease in the propensity to commit other moving violations.

This paper contributes to the literature in several ways. 
It is the first examination using administrative data into the effect of 
an excessive speeding law on driving behaviour 
as proxied by violations by gender and age group. 
Such analysis is important because most countries currently use demerit point systems. 
The question now is not whether these systems work 
but whether and how they can be adjusted to increase road safety. 
Moreover, this paper is to our knowledge the first one to empirically investigate 
the impact of such laws on both the intensive and extensive margins of speeding. 
Finally, by studying the impact of deterrence by gender and age, 
this paper fills a gap acknowledged by 
\citet{freeman1999}
on the role of gender in studies surrounding criminality.


The rest of this article is organized as follows. 
Section \ref{sec:Background} covers the details of Quebec’s excessive speeding law 
and the relevant institutional background. 
The data and summary statistics are presented in Section \ref{sec:Data}. 
% 
We construct a simple theoretical model investigating the effects of the law,
which forms the basis of our empirical specification in Section \ref{sec:Model}. 
% 
In Section \ref{sec:Empirical}, we conduct the empirical analysis. 
A series of robustness and placebo checks is conducted in Section \ref{sec:Validity}. 
We conclude with a policy discussion in Section \ref{sec:Discussion}.

