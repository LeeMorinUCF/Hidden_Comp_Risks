\section{Empirical Results}
\label{sec:Empirical}


\subsection{Regression results for any moving violation}
\label{sec:Empirical_all}

For the regressions in this study, we estimate both the linear probability model and the logistic model. 
%
Due to the very large sample sizes being employed, 
we elect to consider only statistical significance at the 0.1\% and the 0.001\% levels. 
%
Where there is statistical significance at our elevated thresholds, 
the estimates are nearly always quantitatively similar. 
For expositional simplicity, we will focus our interpretation on 
the marginal effects of the linear probability model.%
\footnote{%
For readability, we multiply the estimated coefficients and standard errors by 100,000 
for all tables of regression results for the linear probability model.
% 
For the logistic regression model, 
we calculate the marginal effects of covariate $x_i$ as 
$\beta_{i}\hat{p}(1-\hat{p})$, 
where $\hat{p}$ is calculated as 
the average of the predicted probabilities across the entire relevant sample. 
In the interaction cases, 
we restricted the sample for this calculation as necessary, 
e.g., we restricted the sample to males in the 20--24 age group 
to calculate the marginal effect of the policy indicator for that group. 
}


% Logistic Regression and Linear Probability Models: Seasonal Logit and LPM x 100K Specification for All Drivers by Point Value 

\begin{table}% [ht] 
\centering 
\begin{tabular}{l r r r r l r r l} 

\hline 
 
 & \multicolumn{5}{c}{Logistic Regression}  & \multicolumn{3}{c}{Linear Probability Model} \\ 

 \cmidrule(lr){2-6}\cmidrule(lr){7-9} 
 & \multicolumn{2}{c}{Marginal Effects} & Estimate & Standard & Sig. & Estimate & Standard & Sig. \\ 

 \cmidrule(lr){2-3} 
 &   AME &  MER  &          &  Error   &      &          &  Error   &     \\ 

\hline 
 
\multicolumn{8}{l}{\textbf{Full sample Drivers} (9,675,245,494 observations)} \\ 

\hline
\multicolumn{8}{l}{Model without age-policy interaction: } \\ 
Policy                   &  -3.7849        &  -17.8748       &  -0.0926        &  0.0010       &   **       &  -3.8656        &  0.0411       &   **       \\ 
\hline
\multicolumn{8}{l}{Model with age-policy interaction: } \\ 
Policy                   &  -0.6709        &  -2.9587       &  -0.0435        &  0.0370       &            &  -1.1761        &  0.5700       &            \\ 
Age 16-19 * policy   &  -4.9094        &  -13.7384       &  -0.0684        &  0.0373       &            &  -6.2697        &  0.6707       &   **       \\ 
Age 20-24 * policy   &  -5.3116        &  -14.2921       &  -0.0822        &  0.0371       &            &  -6.7723        &  0.6059       &   **       \\ 
Age 25-34 * policy   &  -3.9081        &  -12.1981       &  -0.0834        &  0.0370       &            &  -5.1489        &  0.5805       &   **       \\ 
Age 35-44 * policy   &  -1.7999        &  -6.0114       &  -0.0430        &  0.0371       &            &  -2.6807        &  0.5780       &   **       \\ 
Age 45-54 * policy   &  -1.1679        &  -4.2241       &  -0.0337        &  0.0371       &            &  -1.9497        &  0.5759       &    *       \\ 
Age 55-64 * policy   &  -0.6156        &  -2.4087       &  -0.0225        &  0.0371       &            &  -1.2160        &  0.5762       &            \\ 
Age 65+ * policy   &  0.7289        &  3.1682       &  0.0385        &  0.0372       &            &  0.3767        &  0.5752       &            \\ 

\hline 

\end{tabular} 
\caption{Pooled Regressions for all offences, male and female drivers} 
For each regression, the dependent variable is an indicator that a driver has committed  
any offence on a particular day.  
All regressions contain age category and demerit point category controls, 
as well as monthly and weekday indicator variables. 
The baseline age category comprises drivers under the age of 16. 
The heading ``Sig.'' is an abbreviation for statistical significance, with 
the symbol * denoting statistical significance at the 0.1\% level 
and ** the 0.001\% level. 
In the linear probability model, coefficients and heteroskedasticity-robust standard errors are  
multiplied by 100,000.  
\label{tab:seas_Logit_vs_LPMx100K_pooled_regs} 
\end{table} 
 


% Logistic Regression and Linear Probability Models: Seasonal Logit and LPM x 100K Specification for All Drivers by Point Value 

\begin{table}% [ht] 
\centering 
\begin{tabular}{l r r l r r l} 

\hline 
 
 & \multicolumn{3}{c}{Logistic Regression}  & \multicolumn{3}{c}{Linear Probability Model} \\ 

 \cmidrule(lr){2-4}\cmidrule(lr){5-7} 
 & Estimate & Std. Error & Sig. & Estimate & Std. Error & Sig. \\ 

\hline 
 
\textbf{Male Drivers} \\ 

\hline
\multicolumn{7}{l}{Model without age-policy interaction: } \\ 
Policy                   &  -0.1113        &  0.0012       &   **       &  -5.9663        &  0.0628       &   **       \\ 
\hline
\multicolumn{7}{l}{Model with age-policy interaction: } \\ 
Policy                   &  -0.0195        &  0.0386       &            &  -1.0915        &  0.7342       &            \\ 
Age 16-19 * policy   &  -0.1107        &  0.0389       &            &  -11.1587        &  0.9191       &   **       \\ 
Age 20-24 * policy   &  -0.1300        &  0.0387       &    *       &  -11.9225        &  0.8017       &   **       \\ 
Age 25-34 * policy   &  -0.1301        &  0.0387       &    *       &  -8.6158        &  0.7536       &   **       \\ 
Age 35-44 * policy   &  -0.0891        &  0.0387       &            &  -5.0295        &  0.7484       &   **       \\ 
Age 45-54 * policy   &  -0.0713        &  0.0387       &            &  -3.5740        &  0.7450       &   **       \\ 
Age 55-64 * policy   &  -0.0594        &  0.0387       &            &  -2.5200        &  0.7455       &    *       \\ 
Age 65+ * policy   &  0.0011        &  0.0389       &            &  -0.2808        &  0.7427       &            \\ 
Observations & \multicolumn{2}{c}{5,335,033,221} \\ 


\hline 

\textbf{Female Drivers} \\ 

\hline
\multicolumn{7}{l}{Model without age-policy interaction: } \\ 
Policy                   &  -0.0294        &  0.0019       &   **       &  -0.8000        &  0.0495       &   **       \\ 
\hline
\multicolumn{7}{l}{Model with age-policy interaction: } \\ 
Policy                   &  -0.0760        &  0.1304       &            &  -0.7470        &  0.6348       &            \\ 
Age 16-19 * policy   &  0.0625        &  0.1307       &            &  0.7804        &  0.7413       &            \\ 
Age 20-24 * policy   &  0.0415        &  0.1305       &            &  -0.0442        &  0.6765       &            \\ 
Age 25-34 * policy   &  0.0200        &  0.1304       &            &  -0.9585        &  0.6483       &            \\ 
Age 35-44 * policy   &  0.0508        &  0.1304       &            &  0.0531        &  0.6458       &            \\ 
Age 45-54 * policy   &  0.0450        &  0.1304       &            &  -0.1831        &  0.6424       &            \\ 
Age 55-64 * policy   &  0.0587        &  0.1305       &            &  0.1339        &  0.6424       &            \\ 
Age 65+ * policy   &  0.1335        &  0.1306       &            &  0.9727        &  0.6416       &            \\ 
Observations & \multicolumn{2}{c}{4,340,212,273} \\ 


\hline 

\end{tabular} 
\caption{Regressions for all drivers (Seasonal Logit and LPM x 100K)} 
All regressions contain age category and demerit point category controls. 
The symbol * denotes statistical significance at the 0.1\% level 
and ** the 0.001\% level. 
``Sig.'' is an abbreviation for statistical significance. 
In the linear probability model, estimates and standard errors are multiplied by 100,000  
and heteroskedasticity-robust errors are employed. 
The baseline age category comprises drivers under the age of 16. 
\label{tab:seas_Logit_vs_LPMx100K_regs} 
\end{table} 
 


We begin by running the regression using the full sample, 
both male and female drivers, 
shown in 
Table \ref{tab:seas_Logit_vs_LPMx100K_pooled_regs}.
% 
In the full sample, we see that the imposition of the policy increases the daily probability 
of receiving a ticket by $0.00387$ percentage points, 
and this is very precisely estimated ($t$-statistic = $-94.054$). 
Once we add the age and policy interactions, 
the coefficient on the policy dummy is no longer significant, 
and we see heterogeneous effects by age group, 
% 
although the significance of the age and policy interactions
depend on the model specification. 
% 
Most of the effect is concentrated in the 16 to 19 and 20 to 24 age groups, 
and then the magnitude declines steadily afterwards. 
The effect loses statistical significance at the 0.1\% level once we reach the 55 to 64 age group.


Although running the pooled regressions in 
Table \ref{tab:seas_Logit_vs_LPMx100K_pooled_regs}
is inconsistent with our theoretical model and our preliminary data analysis, 
we present these results to highlight three issues. 
% 
First, the pooled regressions measure the overall policy effect, 
which the reader may compare to other policy analyses that do not
consider gender differences. 
% 
Second, the misspecification induced by pooling serves as a robustness check
for the interactions by age: 
the age-policy effect is not precisely estimated when specified as 
a proportional change in probability, 
as in the logistic model, 
yet this effect appears to be precisely estimated when specified as 
a constant effect, independent of the other covariates, 
as in the linear probability model.
%
Further, the proportional changes measured by the logistic model 
already account for differences by age group; 
for the linear model, these differences must be accommodated 
by introducing interaction terms.
%
Finally, a nonlinear model measures 
differently the second-order interaction effects, such as the age-policy effect.
% 
The coefficients in the interaction
include terms such as the second cross-partial derivative 
of the probability, with respect to the pair of covariates. 
%
\citet{ainorton2003} demonstrate this and caution that 
 the interaction effect in a logistic model 
is not correctly characterized by the
sign, magnitude, or statistical significance of the coefficient on the
interaction term;
the coefficients in the linear probability model measure the marginal effects
without confounding with the extra terms. 
% 
In summary, these coefficients measure different quantities
and one would not expect them to have the same values 
or pattern of significance. 
% 
Despite these differences, the marginal effects from both models 
in Table \ref{tab:seas_Logit_vs_LPMx100K_pooled_regs}
%
are consistent with each other in that every marginal effect 
from the logistic model is within two standard errors
of the estimates from the linear probability model.
% 
One explanation for the differences between the models 
is that the age differences in policy effect are not as robust
as the differences in policy effects by gender, which we explore next. 


The theoretical model of Section \ref{sec:Model} 
suggests that the effects differ by gender. 
We thus rerun the model, separating the sample by gender, 
the results of which are displayed in 
Table \ref{tab:seas_Logit_vs_LPMx100K_regs}.
In the sample using only males, 
we see that the effect increases the daily probability of receiving a ticket by $0.00597$ 
percentage points, an effect that is approximately 55\% higher than in the pooled sample. 
Once we add the policy and age group interactions, 
the coefficient on the policy dummy again becomes insignificant, 
and the effect appears to be driven by the age groups. 
Examining the coefficients, we see a very distinct pattern: 
the effect is similar between the ages of 16 and 24, 
and the effect begins to decline throughout the entire lifecycle, 
although the effect is no longer significant at the 0.1\% level for the age 65 and over age group.

Conducting the regression for the females in the sample shows that 
the effect is much smaller for this group: 
the effect size of the coefficient on the policy dummy is 13.4\% of the size of its counterpart 
in the regression for males. 
Once the age interactions are added, none of the coefficients are significant at the elevated 1\% level. 
These findings suggest that the pooled results are driven almost entirely by males under the age of 65.
%
That the age-policy interactions are not statistically significant for females 
but are significant for males--- at least for the 20--35 age groups---reinforces 
the notion that the pooled regressions in Table 3 are misspecified. 
Furthermore, the fact that the importance of age-policy interactions for males 
are not fully supported by both regression models suggests that 
age differences are less reliably measured than gender differences.%
\footnote{%
%While the marginal effects of the logistic model for the age-policy interaction variables in 
%Table \ref{tab:seas_Logit_vs_LPMx100K_regs}
%closely follow those of the linear probability model, they are less precisely estimated. 
%One reason this could be the case is that the interaction effect in a logistic model 
%is not correctly characterized by the
%sign, magnitude, or statistical significance of the coefficient on the
%interaction term 
%%
%\citep{ainorton2003};
%the linear probability model does not have this limitation.
% 
Again, we refer the reader to \citet{ainorton2003}
for an explanation of the differences in significance in the age-policy interactions between the logistic regression model and the linear probability model. 
}
%

It is important to note that the estimate of the effect of the law in the main regression 
can be interpreted as an average treatment effect; 
this treatment effect includes drivers who rarely sufficiently exceed the speed limit 
or otherwise break the law to be penalized with traffic tickets. 
Assuming these more careful drivers are not affected by the law at all 
and that they make up a large segment of the population, 
the effect of the law on the relevant subpopulation that is affected by the law 
may be well underestimated.%
\footnote{%
Whether to interpret these estimates as average treatment effects 
is a question that has not yet been broached in the literature. 
We briefly consider this issue here. 
Since the entire population is being treated by the policy change, 
one can argue that the average treatment effect (ATE) equals 
the average treatment effect on the treated (ATT). 
However, one may claim that since the law was only meant to catch people 
who routinely speed in the first place, 
this subpopulation of habitual speeders make up the treatment group 
and thus the average effect on them would be the ATT, 
while the ATE refers to the average effect on the whole population.
}
%


\subsection{Regression results by point total}
\label{sec:Empirical_by_pts}

In this section, we examine the effects of Quebec’s excessive speeding law by point total. 
We repeat the policy dummy specification in 
Section \ref{sec:Empirical_all} 
but run a regression for each particular ticket point value: 1, 2, 3, 4, 5, 7, and 9 or more points. 
For each of these regressions, the dependent variable is equal to 1 
if the driver earns a ticket of that point value on that day, and is equal to 0 otherwise. 
This strategy will allow us to investigate the changes in the intensive margin of 
demerit points given to drivers after the policy change. 
Individuals may substitute driving well above the speed limit with driving at lower speeds 
but still above the speed limit. 
As before, the demerit points lost after the policy change take into account 
the doubling of the penalty due to the excessive speeding law. 
For example, the 5-point category therefore includes tickets 
worth 5 points before the policy change and 5 or 10 points after the policy change. 
These effects might be slightly underestimated (that is, they may have a slight downward bias) 
since some ticket combinations yielding 10 points after the policy change 
would be captured by these regressions. 
However, as previously argued, these sorts of incidents are likely very rare. 


% Logistic Regression and Linear Probability Models: Seasonal Logit and LPM x 100K Specification by Point Value 

\begin{table}% [ht] 
\centering 
\begin{tabular}{l r r r l r r l} 

\hline 
 
 & \multicolumn{4}{c}{Logistic Regression}  & \multicolumn{3}{c}{Linear Probability Model} \\ 

 \cmidrule(lr){2-5}\cmidrule(lr){6-8} 
 & Marginal & Estimate & Standard & Sig. & Estimate & Standard & Sig. \\ 
 &   Effect &          &  Error   &      &          &  Error   &     \\ 

\hline 
 
\multicolumn{7}{l}{\textbf{Male Drivers} (5,335,033,221 observations)} \\ 

All point values                &  -2.8074       &  -0.1113        &  0.0012       &   **       &  -5.9663        &  0.0628       &   **       \\ 
1 point                         &  0.6567       &  0.0953        &  0.0043       &   **       &  0.3930        &  0.0177       &   **       \\ 
2 points                        &  0.7076       &  -0.0191        &  0.0019       &   **       &  -0.4315        &  0.0394       &   **       \\ 
3 points                        &  -3.2138       &  -0.1872        &  0.0017       &   **       &  -4.7786        &  0.0436       &   **       \\ 
4 points                        &  -0.0319       &  -0.1252        &  0.0114       &   **       &  -0.0804        &  0.0066       &   **       \\ 
5 points                        &  -0.7175       &  -0.6470        &  0.0080       &   **       &  -0.8189        &  0.0100       &   **       \\ 
7 points                        &  -0.1403       &  -0.7392        &  0.0193       &   **       &  -0.1625        &  0.0042       &   **       \\ 
9 or more points                &  -0.0493       &  -0.2501        &  0.0170       &   **       &  -0.0675        &  0.0045       &   **       \\ 

\hline 

\multicolumn{7}{l}{\textbf{Female Drivers} (4,340,212,273 observations)} \\ 

All point values                &  0.3996       &  -0.0294        &  0.0019       &   **       &  -0.8000        &  0.0495       &   **       \\ 
1 point                         &  0.6190       &  0.2124        &  0.0062       &   **       &  0.5174        &  0.0150       &   **       \\ 
2 points                        &  0.8602       &  0.0303        &  0.0028       &   **       &  0.3613        &  0.0336       &   **       \\ 
3 points                        &  -0.8653       &  -0.1256        &  0.0029       &   **       &  -1.4289        &  0.0323       &   **       \\ 
4 points                        &  0.0057       &  -0.0098        &  0.0293       &            &  -0.0010        &  0.0032       &            \\ 
5 points                        &  -0.1896       &  -0.7494        &  0.0187       &   **       &  -0.2105        &  0.0053       &   **       \\ 
7 points                        &  -0.0169       &  -0.9113        &  0.0695       &   **       &  -0.0191        &  0.0015       &   **       \\ 
9 or more points                &  -0.0125       &  -0.1541        &  0.0282       &   **       &  -0.0180        &  0.0033       &   **       \\ 

\hline 

\end{tabular} 
\caption{Regressions by ticket-point value} 
In each row, the dependent variable is an indicator that a driver has committed  
an offence with the stated point value on a particular day.  
The categories of tickets with 3, 5 and 7 points includes tickets  
with 6, 10 and 14 points after the policy change, respectively,  
and the category with 9 or more points includes tickets  
with all corresponding doubled values after the policy change. 
All regressions contain age category and demerit point category controls, 
as well as monthly and weekday indicator variables. 
The baseline age category comprises drivers under the age of 16. 
The heading ``Sig.'' is an abbreviation for statistical significance, with 
the symbol * denoting statistical significance at the 0.1\% level 
and ** the 0.001\% level. 
In the linear probability model, coefficients and heteroskedasticity-robust standard errors are  
multiplied by 100,000.  
\label{tab:seas_Logit_vs_LPMx100K_regs_by_points} 
\end{table} 
 



We see the results of these regressions by ticket point value in 
Table \ref{tab:seas_Logit_vs_LPMx100K_regs_by_points}. 
For males, we see a very minor increase in the number of tickets 
worth 1 point after the policy change. 
This increase in 1-point tickets is dwarfed by the decrease in the tickets 
in all of the other point categories and is alone cancelled out 
by the decrease in 2-point tickets. 
For females, a similar pattern is found in that 1- and 2-point tickets increase slightly,
but this increase is more than cancelled out by the decrease in 3-point tickets. 
There is a decrease in 4-point tickets, but it is not precisely estimated. 
All ticket values of 5 or more points decrease after the policy change. 
Note that the coefficient sizes for some of the higher ticket point categories on 
Table \ref{tab:seas_Logit_vs_LPMx100K_regs_by_points}
are quite small: 
this is because high ticket values are rare, 
and so any decrease in them will have a smaller coefficient, 
since it represents a change from one small number to another small number.

These patterns suggest that many drivers have decreased their maximum speed 
after the policy change. 
It appears likely that many people who used to speed well above the limit 
have decreased their speed such that they are still exceeding the limit, 
but not as much as before. 
Since the extensive margin of tickets has decreased, 
it looks like many who used to speed at moderate speeds over the limit 
no longer exceed the speed limit.


\subsection{Regression results for drivers with high point balances}
\label{sec:Empirical_high_pts}

It may be of interest to know how drivers who typically drive less carefully 
(and thus accumulate more demerit points) 
may have seen their point balances shift on average after the implementation of the policy. 
We examine the subsample of drivers who at one point in the pre-period 
had a point balance of between 6 and 10 demerit points 
using the regression specification of 
Section \ref{sec:Empirical_all}. 
Therefore, two categories of drivers are excluded: 
those whose point balance never reaches 6 (most of the sample), 
and those who received serious tickets and therefore whose point balance is never in this range. 
For example, a person who received a singular ticket for excessive speeding worth 12 demerit points 
will not be a part of this sample because their point balance will remain at 12 
as long as the ticket is on their record, 
and the balance will drop down to 0 when the ticket’s demerit points expire: 
at no point was this driver’s demerit point balance between 6 and 10. 
We need to exclude these drivers to avoid issues associated with the drivers’ licence revocation. 
Indeed, a revocation would necessarily lead to a reduction in the number of violations 
in the post-policy period, because the individual is not allowed to drive. 
The results of this exercise by gender are on 
Table \ref{tab:seas_Logit_vs_LPMx100K_high_pt_regs_by_points} below. 

% Logistic Regression and Linear Probability Models: Seasonal Logit and LPM x 100K Specification for High-Point Drivers by Point Value 

\begin{table}% [ht] 
\centering 
\begin{tabular}{l r r l r r l} 

\hline 
 
 & \multicolumn{3}{c}{Logistic Regression}  & \multicolumn{3}{c}{Linear Probability Model} \\ 

 \cmidrule(lr){2-4}\cmidrule(lr){5-7} 
 & Estimate & Std. Error & Sig. & Estimate & Std. Error & Sig. \\ 

\hline 
 
\multicolumn{7}{l}{\textbf{Male Drivers} (921,131,812 observations)} \\ 

All point values                &  -0.3732        &  0.0021       &   **       &  -38.0770        &  0.2114       &   **       \\ 
1 points                        &  -0.0735        &  0.0076       &   **       &  -0.5454        &  0.0572       &   **       \\ 
2 points                        &  -0.2111        &  0.0035       &   **       &  -7.7125        &  0.1261       &   **       \\ 
3 points                        &  -0.4677        &  0.0029       &   **       &  -24.5075        &  0.1520       &   **       \\ 
4 points                        &  -0.8975        &  0.0228       &   **       &  -0.8445        &  0.0205       &   **       \\ 
5 points                        &  -1.0016        &  0.0124       &   **       &  -3.3206        &  0.0393       &   **       \\ 
7 points                        &  -1.1495        &  0.0291       &   **       &  -0.7270        &  0.0173       &   **       \\ 
9 or more points                &  -0.7647        &  0.0319       &   **       &  -0.3543        &  0.0145       &   **       \\ 

\hline 

\multicolumn{7}{l}{\textbf{Female Drivers} (249,294,614 observations)} \\ 

All point values                &  -0.4252        &  0.0052       &   **       &  -26.0411        &  0.3154       &   **       \\ 
1 points                        &  -0.0239        &  0.0193       &            &  -0.0916        &  0.0830       &            \\ 
2 points                        &  -0.2441        &  0.0082       &   **       &  -5.9044        &  0.1970       &   **       \\ 
3 points                        &  -0.5749        &  0.0075       &   **       &  -17.6976        &  0.2250       &   **       \\ 
4 points                        &  -1.2986        &  0.1060       &   **       &  -0.2424        &  0.0181       &   **       \\ 
5 points                        &  -1.3612        &  0.0425       &   **       &  -1.6387        &  0.0469       &   **       \\ 
7 points                        &  -1.6962        &  0.1444       &   **       &  -0.2020        &  0.0151       &   **       \\ 
9 or more points                &  -1.1624        &  0.0942       &   **       &  -0.2568        &  0.0202       &   **       \\ 

\hline 

\end{tabular} 
\caption{Regressions for high-point drivers by ticket-point value (Seasonal Logit and LPM x 100K)} 
All regressions contain age category and demerit point category controls. 
The symbol * denotes statistical significance at the 0.1\% level 
and ** the 0.001\% level. 
``Sig.'' is an abbreviation for statistical significance. 
In the linear probability model, estimates and standard errors are multiplied by 100,000. 
Heteroskedasticity-robust errors are employed. 
The baseline age category comprises drivers under the age of 16. 
The 5 point category of tickets includes 10 point tickets after the policy change,  
the 7 point category includes 14 point tickets after the policy change,  
and 9 or more point tickets include all possible doubled values for those tickets  
worth more than 9 points after the policy change. 
\label{tab:seas_Logit_vs_LPMx100K_regs_by_points} 
\end{table} 
 



For both males and females, 
the effect of the policy both in general and by ticket point value 
shows much larger effects in the negative direction. 
For example, the effect of the policy on males for 3-point tickets is five times larger 
in the high point group compared to the overall sample. 
Overall, the frequency of tickets decreases by a relatively large margin 
for this group of drivers after the policy.




