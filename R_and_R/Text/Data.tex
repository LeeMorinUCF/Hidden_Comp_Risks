\section{Data}
\label{sec:Data}


We use records of traffic violations and drivers licenses obtained from SAAQ administrative data to generate a dataset containing the universe of driver-days from April 1st 2006 to March 31st 2010 for the province of Quebec.\footnote{%
The dataset on driver’s licenses allows us to include observations that do not receive any tickets during the sample period.}  
Our dataset contains information on the age, gender, and details concerning traffic violations of the offender. In all, we have approximately 9.7 billion driver-day observations over the sample period. This very large sample will afford us the opportunity to examine detailed subgroups and give us the statistical power to detect effects that are small in absolute magnitude.

We begin with a graphical analysis of some select demerit point values. Here, we examine monthly ticket frequencies for given point values before and after the policy change. Unfortunately, the dataset makes it impossible to distinguish between single and multiple violations for a single police stop. For example, it is impossible to distinguish a driver with two 3-point violations from a driver with a single 6-point violation – all we observe is that both drivers gained 6 demerit points on a given day. Fortunately, multiple violation stops are likely very rare in our sample.\footnote{%
For example, before the excessive speeding law, there were no violations worth 6 points, but the sample shows 517 stops resulting in 6 demerit points compared to 43,006 stops resulting in 5 demerit points. As another example, a single 7-point violation was present before the policy change, but none after; the number of 7-point tickets before the policy change was 8,366, and it decreased to only 24 after the policy change. There are no violations in the Highway Safety Code worth 8 or 11 demerit points at any time in our sample period, and our data shows no stops with demerit point totals of these values.}

Since the demerit point values of some violations have doubled after the excessive speeding law came into effect under certain conditions (see Section 2 for details), we will compare stops associated with a certain number of points before the policy change with those associated with the same number of points and double the number of points after the policy change. For example, a driver speeding 46km/h to 49km/h over the speed limit before the policy change would receive a 5-point ticket, but the same violation would be worth 10 points after the policy change if it qualifies for excessive speeding. Because excessive speeding doubles the point values of some speeding violations, we will need to compare the frequency of 5-point stops before the policy change to 5- or 10-point stops afterwards (as not all 5-point speeding violations may qualify as excessive speeding). Due to the aforementioned possibility of stops with multiple violations, the number of tickets with 5- or 10-point values will contain combinations of violations which will be counted in the post period that were not counted in the pre period, and so the effect of the law will be underestimated in this case, but the overall effect should be minimal.

If drivers do not adjust their behaviour, there should be about as many drivers with 5 points before the policy as there were drivers with 5 or 10 points after the policy. If drivers slow down, the number of 5-point or 10-point violations will decrease. \\

\textbf{Figure 1 about here} \\

Taking into consideration the seasonality of speeding, we see an important reduction in the number of 5- or 10-point tickets in the summer of 2008 than in the preceding summer. Overall, there is a general downward trend in the number of tickets after the policy change compared to before the policy change, and the 5- and 10-point after the change are approximately evenly split. 
With 7- or 14-point stops, we see a different picture: nearly all 7-point violations are worth 14 points after the policy change, while only a few 7-point violations remain. Since there is no violation worth 7 points after the policy change, all of the 7-point stops after the policy change are due to being pulled over for multiple violations totalling 7 points. Once again, we see a downward trend in the number of total violations. \\

\textbf{Figure 2 about here} \\

Table 2 reports summary statistics. Overall, since females represent half of drivers yet only 20\% of traffic violations, females represent a far smaller share of those who receive tickets compared to males. In the 1- and 2-point categories, the number of tickets increases for both males and females on a per driver-day basis, and generally decreases in the higher-point categories, even though more violations lead to an increase of points.

To put these numbers in a broader context, the vast majority of the sample are non-events. Before the excessive speeding law came into effect, the average driver has a probability of 0.04\% to receive a ticket on any particular day. This probability decreases by approximately 3.6\% after the policy change. If we look at the demerit points per driver per day, they decreased after the policy change for males by 6\% and by 1\% for females. This result is particularly interesting, because excessive speeding penalties doubled the value of many speeding violations previously worth 5, 7, and 9 points.\footnote{%  
Some 3-point speeding tickets are subject to the excessive speeding law, but the circumstances are quite particular: the suspect needs to be exceeding the speed limit in a zone with a posted limit of 60km/h or less by 40 to 45 km/h.}
In the absence of a change in behaviour, the number of demerit points per driver per day would have mechanically increased. \\

\textbf{Table 2 about here} \\

