\section{Policy Discussion}
\label{sec:Discussion}


This paper studies the impact of an increase in penalties for speeding. We find a decrease in the number of severe violations, especially for young males. These results offer some important policy implications.

First, we document a significant spillover effect from the introduction of the law. Even though excessive speeding events targeted by the law are relatively rare, the law appears to have influenced many drivers. Indeed, if the only reactions were from people usually speeding well above the speed limit, we would likely not find statistically detectable effects in the overall sample, given the low share of tickets for excessive speeding (see Table 2). Overall, the legislation decreased the number of violations and caused people who habitually drive above the limit to speed less. This spillover effect is somewhat puzzling. Perhaps the excessive speeding law made people more aware of speeding penalties in general. Since drivers do not always pay close attention to their speed, the threat of more severe penalties may have increased their awareness for their speed in more mundane circumstances.

Second, this analysis shows variations in the capacity to deter different groups of individuals. Young males seem to react particularly strongly to such penalties. Since this group is usually responsible for dangerous driving, this type of policy seems to be particularly effective to improve roadway safety.


%\pagebreak
%Gratuitous list of citations goes here: 
%
% \citet{tardiff2010}
% \citet{shi2009}
% \citet{levittmiles2006}
% \citet{levitt1997}
% \citet{klicktabarrock2005}
% \citet{hussetal2006}
% \citet{hellandtabarrok2007}
% \citet{hansen2015}
%\citet{haque1990}
% \citet{evansowens2007}
% \citet{dmw2011}
% \citet{ditellaschar2004}
% \citet{dionneetal2011}
% \citet{deangelohansen2014}
% \citet{bourgeonpicard2007}
% \citet{bennedittiniNicita2009}
% \citet{bjm2014}
%% \citet{becker1968}
% \citet{becker1968a}
% \citet{becker1968b}
% \citet{pulido2010}