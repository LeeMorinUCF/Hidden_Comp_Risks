\section{Empirical Analysis}
\label{sec:Empirical}


\subsection{Regression results for any moving violation}

We analyze the effect of the excessive speeding law on traffic tickets by means of an event study using a linear probability model. The main regression specification is
\begin{equation}
y_{it} = \beta_0 + \beta_1 d_t
      + \beta_2 d_t agecat_{it} + \beta_3 agecat_{it} + \beta_4 ptsgrp_{it}
      + \epsilon_{it}
\end{equation}

where $d_t$ is a dummy variable equal to 1 after the policy change and 0 before, $agecat$ is a set of age category dummies, $ptsgrp$ is a set of demerit point balance categories, and $\epsilon_it$ is the usual error term.\footnote{% 
Note that the bolded items represent vectors rather than scalars.
}  
The dependent variable y is equal to 1 if the individual received a ticket on that day and 0 otherwise. The age category controls are 16 to 19, 20 to 24, 25 to 34, 35 to 44, 45 to 54, 55 to 64, and 65 and over. The demerit point balance is the sum of demerit points on a driver’s record over the last two years. This variable is divided in to categories of 1 to 3, 4 to 6, 7 to 9, and 10 and over.

Our coefficients of interest are the scalar $\beta_1$ and the vector $\beta_2$. We include the vector $\beta_2$ in some specifications and run some regressions separating by gender because the theoretical model of Section 3 predicted the possibility of heterogeneous effects due to differential attitudes towards risk: females and those of higher ages are likely to be more risk averse. \\

\textbf{Table 3 about here} \\

The regression results are displayed above in Table 3. Due to the very large sample sizes being employed, we elect to only consider statistical significance at the 0.1\% and the 0.001\% level. In the full sample, we see that the imposition of the policy increases the daily probability of receiving a ticket by 0.00382 percentage points, and this is very precisely estimated (t-statistic = -92.992). Once we add the age and policy interactions, the coefficient on the policy dummy is no longer significant, and we see very heterogeneous effects by age group. Most of the effect is concentrated in the 16 to 19 and 20 to 24 age groups, and then the magnitude declines steadily afterwards. The effect loses statistical significance at the 0.1\% level once we reach the 45 to 54 age group.

The theoretical model of Section 3 suggests that the effects differ by gender. We rerun the model using only males and see that the effect increases the daily probability of receiving a ticket by 0.00592 percentage points, an effect 55\% higher than in the pooled sample. Once we add the policy and age group interactions, the coefficient on the policy dummy again becomes insignificant, and the effect appears to be driven by age groups. Examining the coefficients, we see a very distinct pattern: the effect is more or less constant between the ages of 16 and 24, and the effect begins to decline throughout the entire lifecycle, although the effect is no longer significant at the 0.1\% level at the age 55 to 64 age group.\footnote{% 
Be mindful of the change in the exponents in scientific notation as the effect weakens with age, e.g. such as the step from the 20 to 24 age group to the 25 to 34 age group.
}

Conducting the regression for the females in the sample shows that the effect is much smaller for this group: the effect size of the coefficient on the policy dummy is 12.8\% of the size of its counterpart in the regression for males. Once the age interactions are added, none of the coefficients are significant at the elevated 1\% level. These findings suggest that the pooled results are driven almost entirely by males under the age of 55.

It is important to note that the estimate of the effect of the law in the main regression is an average treatment effect; this treatment effect includes drivers who rarely sufficiently exceed the speed limit or otherwise break the law to be penalized with traffic tickets. Assuming these more careful drivers are not affected by the law at all and that they make up a large segment of the population, the effect of the law on the relevant subpopulation that is affected by the law may be well underestimated.


\subsection{Regression results by point total}

In this section, we examine the effects of Quebec’s excessive speeding law by point total. We repeat the policy dummy specification in section 5.1 but restrict the sample to tickets worth 1, 2, 3, 4, 5, 7, and 9 or more points. This strategy will allow us to investigate the changes in the intensive margin of demerit points given to drivers after the policy change. Individuals may substitute driving well above the speed limit with driving at lower speeds but still above the speed limit. As before, the demerit points lost after the policy change take into account the doubling of the penalty due to the excessive speeding law. For example, the 5-point category therefore includes tickets worth 5 points before the policy change and 5 or 10 points after the policy change. These effects might be slightly underestimated (that is, they may have a slight downward bias) since some ticket combinations yielding 10 points after the policy change would be captured by these regressions. However, as previously argued, these sorts of incidents are likely very rare. \\

\textbf{Table 4 about here} \\

We see the results of these regressions by ticket point in Table 4. For males, we see a very minor increase in the number of tickets worth 1 point after the policy change. This increase in 1-point tickets is dwarfed by the decrease in the tickets in all of the other point categories and is alone cancelled out by the decrease in two point tickets. For females, a similar pattern is found in that 1- and 2-point tickets increase slightly, but this increase is more than cancelled out by the decrease in 3-point tickets. There is a decrease in 4-point tickets, but it is not precisely estimated. All ticket values of 5 or more points decrease after the policy change. Note that the coefficient sizes for some of the higher ticket point categories on Table 4 are quite small: this is because high ticket values are rare, and so any decrease in them will have a smaller coefficient, since it represents a change from one small number to another small number.

These patterns suggest that many drivers have decreased their average speed after the policy change. It appears likely that many people who used to speed well above the limit have decreased their speed such that they are still exceeding the limit, but not as much as before. Since the extensive margin of tickets has decreased, it looks like many who used to speed at moderate speeds over the limit are now no longer speeding.


\subsection{Regression results for drivers with high point balances}

It may be of interest to know how drivers who typically drive less carefully (and thus accumulate more demerit points) may have seen their point balances shift on average after the implementation of the policy. We examine the subsample of drivers who at one point in the pre-period had a point balance of between 6 and 10 demerit points. Therefore, two categories of drivers are excluded: those whose point balance never reaches 6 (most of the sample), and those who received serious tickets and therefore whose point balance is never in this range. For example, a person who received a singular ticket for excessive speeding worth 12 demerit points will not be a part of this sample because their point balance will remain at 12 as long as the ticket is on their record, and the balance will drop down to 0 when the ticket’s demerit points expire: at no point was this driver’s demerit point balance between 6 and 10. The results of this exercise by gender are on Table 5 below. \\

\textbf{Table 5 about here} \\

For males, the effect on high point drivers has an interesting pattern by age. The effect is very strong for people age 16 to 19, decreases in the next age group, but then is more or less steady after that. For women, the policy has much less of an effect than for males, and the effect also smoothly decreases with age. Overall, the frequency of tickets decreases by a relatively large margin for this group of drivers after the policy.




