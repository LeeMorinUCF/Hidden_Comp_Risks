\section{Concerns of Validity}
\label{sec:Validity}


\subsection{Alternative explanations for the downturn in tickets}

To rule out any possibility that these results are driven by secular trend, we repeat the regression specified in Section 5.1 by splitting the pre-period in half. Because of the very heterogeneous effects by gender, we perform two sets of placebo checks, one for males and one for females. The results of this analysis are displayed in Table 5. \\

\textbf{Table 6 about here} \\

The regression results show no evidence of pre-trends. In the regressions without the policy and age interactions, the coefficients are not precisely estimated. Moreover, the magnitude of the coefficients in both placebo regressions are almost identical and their magnitude corresponds to approximately one-eighth of those in the real regression for males. If there were a secular trend in the pre-period driving the results of the main regressions, the male coefficient would have a larger magnitude than the female coefficient, but this is not the case. In the set of regressions containing the age category dummies interacted with the policy variable, none of the interactions are statistically significant, and there is no pattern among the coefficients either. This result contrasts with the coefficients on the age interactions for males in the main regression in which we observe a clear pattern: the effect is similar from ages 16 to 25, and then slowly declines with age. Overall, there is no evidence that the effects found in the real regression are an artifact of something other than the excessive speeding laws.

During about the last third of the sample period after the policy change, the province of Quebec introduced a photo radar pilot, which started to distribute tickets for speeding on August 19th, 2009. We do not believe this had a substantial effect on driver behaviour for several reasons. First, the number of photo radar machines was very small: there were 15 in the entire province of Quebec, which had a population of approximately 8 million people at the time.\footnote{% 
Of these 15, 6 were for speeding, 6 were for red light violations, and 3 were mobile 
%(La Presse, 2020).
\citet{bisson2020}.
}  
Second, during the pilot, the photo radar machines were placed in plain sight, and warning signs were placed ahead of them to clearly alert drivers of their location.

An alternative explanation is the idea that police leniency may have changed as a result of the law. We do not believe this is the case. One can assume that the introduction of additional penalties for excessive speeding may motivate police officers to write fewer tickets for these crimes and note them as lesser speeding violations instead, under the assumption that they have a distaste for appearing in court. In Quebec, however, police officers are not required to appear in traffic court.\footnote{% 
See the following website (in French) for details: \texttt{https://educaloi.qc.ca/capsules/la-contestation-dune-contravention/} (Accessed July 18th, 2020).}  
In terms of police behaviour specifically in response to those who speed at a level to obtain excessive speeding penalties under the new regime, it would be reasonable to assume that a police officer aiming to be lenient would reduce the speed on a ticket to a level where the excessive speeding provisions would not take effect. For example, excessive speeding in zones with a limit of 60km/h could be marked down to a 3 point ticket, while excessive speeding in zones with higher limits could be reduced to a 5 point ticket. 

However, according to Table 4, the incidence of tickets for men decreases for every category above 1-point, while for women it decreases for every category above 2 points, and so the categories to which the tickets would be marked down still saw decreases.\footnote{% 
Note that there are no speeding violations worth 4 points under the Quebec highway safety code, and that 4-point tickets are much less common than 3-point or 5-point tickets, see Table 2.
}  
If driving behaviour had not changed and police had become more lenient by marking down speeds below the excessive speeding thresholds, we would see an increase in tickets for the 3 and 5 point categories, which we do not observe in the data. Finally, the overall number of tickets per driver day still decreased (the extensive margin), and leniency against the provisions of the excessive speeding law would only affect the intensive margin of demerit points. We conclude it is very unlikely that a change in police leniency could be driving the results.


\subsection{Statistical properties}

Concerns may be raised about the mathematical properties of estimates derived from the use of linear probability models. For example, 
% Horrace and Oaxaca (2006) 
\citet{horraceoaxaca2006}
claim that predicted probabilities outside of the $[0,1]$ interval are indicative of bias and inconsistency of the linear probability model regression estimates. For all regressions conducted in our paper, no predicted probabilities fall outside of this interval, thus ruling out this possibility. Furthermore, the absence of negative predictions is not a product of chance; the explanatory variables in our regressions are all categorical variables. The predictions are essentially proportions, rather than linear predictions from continuous variables, and this mitigates the usual criticism of the linear probability model. 

Another issue is the relative rarity of the events (the driver-days where the dependent variable is equal 1 rather than 0). 
% King and Zeng (2001) 
\citet{kingzheng2001}
show that rare events cause estimated probabilities to be biased downwards for logit estimation (in the case where ones are rare relative to zeros). The level of the rare events bias is a function of the frequency of events relative to the total sample size: for example, a sample size of 1,000 with 2 events (0.2\% of the sample) may suffer from rare events bias, but a sample size of 100,000 with 200 events (also 0.2\%) may not. To examine whether rare events bias potentially exists in our analysis, we conduct a simulation as follows. We set up a simulation using an effect size that is similar to the regression involving females but uses a much smaller sample size: if rare events bias is absent, it should be absent in the real regression which has a sample 100 times larger. The simulation has 1,000 repetitions. For each, we generate a dataset with 43,390,582 observations where 0.00369\% of observations in the pre-period have an event, and 0.004449\% in the post-period. The effect size of interest is the difference between these two numbers which is 0.000759\%. The results are as follows. We find no evidence of rare events bias: the mean effect size of the simulations is also 0.000759\% and the estimates are tightly distributed, with the 25th percentile being equal to 0.000620\%, and the 75th percentile to 0.000889\%. Moreover, the statistical power is healthy, with 30.1\% of the samples producing statistically significant results for the effect size at the 0.001\% level. This is despite the simulation using a sample size only one one-hundredth that of the sample used in the analysis. We conclude that there is no evidence that rare events bias is influencing the results.

