\section{Theoretical Model}
\label{sec:Model}

In this section, we present a simple theoretical model of driving behaviour. 
% ... and use it to guide our empirical specification. 
We use this to appeal to economic theory to determine whether to expect 
differences in age and gender as a result of a policy that increases the risk of driving, 
and if so, whether there would be a pattern in these differences. 
% 
We later use the conclusions reached from this analysis 
as predictions of the results of the empirical model.
% ... i.e. we use it to guide our empirical specification
% and provide testable predictions for the empirical analysis.  

Consider the utility maximization problem 
for the representative agent
%
% \begin{equation}
$$
	u_j (s) = g(s) - r_j (s)
$$
% \end{equation}
% 
where $g(s)$ is the utility of driving at speed $s$ 
and $r_j (s)$ is the disutility from the risk of driving at speed $s$, 
and $j$ indexes males and females $\{m,f\}$;
% 
therefore, we assume the representative male and the representative female 
have different risk preferences (and therefore utility functions). 
% 
Assume $g(s)$ is concave increasing $(g^{\prime} (s)>0, 
g^{\prime\prime} (s)<0)$ 
and $r_j (s)$ is convex increasing $(r_j^{\prime} (s)>0, r_j^{\prime\prime} (s)>0)$. 
Let $g(s)$ and $r_j (s)$ be continuous in the positive orthant. 
Impose the regularity conditions $g(s)\geq0 \forall s$ and $r_j (s)\geq0 \forall s$. 
Let there exist values of s such that $g(s)>r_j (s)>0$; 
this guarantees the existence of a non-trivial equilibrium. 
Taking the first order condition of the objective function, 
the ideal speed $s^*$ is chosen such that $g^{\prime} (s)=r_j^{\prime} (s)$, 
and this is a global maximum because 
$u_j^{\prime\prime} (s)=g^{\prime\prime} (s)-r_j^{\prime\prime} (s)<0$. 
Plotting each curve separately on a graph, 
the maximum point $s^*$ is the one such that 
the vertical distance between the concave and convex curves is maximized, 
and this occurs at the point where the slopes are equal. 
Let $r_m (s)<r_f (s)  \forall s$; 
that is, the perceived risk of driving at any given speed 
is higher for females than it is for males 
% (Croson and Gneezy, 2009). 
\citep{crosongneezy2009}.
Graphically, the risk function for the representative females will be more convex than it is for the representative male. 
Examining the first order conditions, 
we see that, on average, males will drive faster than females ($s_m^*>s_f^*$) 
since 
$u_m^{\prime} (s)
=g^{\prime} (s)-r_m^{\prime} (s)
>g^{\prime} (s)-r_f^{\prime} (s)
=u_f^{\prime} (s)$. \\


% Jeff says:
% I think we should set the proof in standard mathematical emphasis, i.e. the word "Proposition 1." in bold, the proposition itself in italics, and the word "Proof." in bold.
% In later drafts, I will use the proof environment.

\textbf{Proposition 1.} {\it Suppose the risk profile increases 
for both males and females such that driving at speed $s$ 
produces a risk of $r_j (s+\epsilon)$. 
Then, the decrease in driving speed for males will be greater 
than the decrease in driving speed for females.} \\

\textbf{Proof:} Let the new equilibrium point be labelled $s_j^{**}$. 
It is immediate that $s_j^*>s_j^{**}$ 
for both $j=\{m,f\}$ by the convexity of $r_j (s)$. 
By the concavity of $g(s)$ 
and because 
$r_m (s)<r_f (s)  \forall s, 
(s_m^*-s_m^{**} ) - (s_f^*-s_f^{**})>0$. \\

Informally, the female objective function for the representative female 
will reach its new equilibrium speed sooner 
because both $g(s)$ and $r_j (s)$ are steeper 
when moving from the old equilibrium to the new equilibrium.

This theoretical model predicts that people who are more risk averse 
towards a certain behaviour are less likely to be affected 
by additional disincentives for that behaviour. 
If the penalties for speeding increase, 
females are less likely to be affected because they are more risk averse. 
The model can analogously be applied to age: 
younger people tend to be more risk seeking 
% (e.g. Gong and Yang, 2011), 
(e.g. \citet{gongyang2012}), 
so we also predict that our empirical results will show that the effect of the law 
on risk taking behaviour in driving will decrease with age.
