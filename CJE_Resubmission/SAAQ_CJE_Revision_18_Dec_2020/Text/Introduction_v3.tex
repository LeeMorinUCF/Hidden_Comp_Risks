\section{Introduction}
\label{sec:Introduction}

In 2018, 1,743 individuals died in Canada following a car accident 
% (Transport Canada, 2020). 
\citep{transcan2018}. 
Such accidents are the leading cause of death for individuals aged between 15 and 44
% (Statistics Canada, 2020).  
\citep{statscan2020}.   
Many OECD countries have recently introduced harsher punishments to deter 
the types of behaviour that increase the likelihood of such tragedies. 
For example, penalties for speeding well above the speed limit now typically include 
some combination of substantially increased fines, immediate vehicle seizure, and license suspension.
These laws are typically referred to as excessive speeding laws or stunt-driving laws. 

Quebec followed this trend and introduced excessive speeding penalties in 2008. 
Its provisions are triggered when driving well above the speed limit. 
For example, driving at a speed of 100km/h in a 60km/h zone would be considered excessive speeding.
These harsher punishments received widespread media coverage 
both before and after its implementation, 
and there was a sustained campaign by the provincial government 
to help ensure drivers were aware of the law. 
The Quebec government has since declared the legislation to be successful, 
with the number of excessive speeding tickets decreasing over time. 
Furthermore, the number of accidents with bodily harm 
decreased from 36,816 in 2006 to 32,371 in 2010, 
and the number of accidents with fatalities decreased from 666 to 441 
over the same period 
% (SAAQ, 2011). 
% \citep{saaq2011}. 
(\citet{saaq2011}). 
Even though these findings are encouraging, 
they don’t provide conclusive evidence that the harsher penalties truly caused changes in behaviour. 
The legislation has been unsuccessfully challenged in court.

Since 
% Becker (1968), 
\citet{becker1968b}, 
economists have theorized that harsher punishments 
alter the incentives of individuals and thus ultimately their behaviour. 
% Helland and Tabarrok (2007) 
\citet{hellandtabarrok2007} 
provide evidence for this mechanism 
studying the impact of California’s three-strike legislation on the recidivism rates of felons. 
However, the effectiveness of deterrence is unclear in the context of driving. 
Indeed, 
% Bourgeon and Picard (2007) 
\citet{bourgeonpicard2007} 
hypothesize that some drivers may be impossible to deter 
either because they do not care about the penalties or because they are not aware they are speeding.

{\Large Talk more about the Meirambayeva paper.
We extend it in many ways.}

A broad literature has investigated the role of deterrence on driving 
and particularly alcohol consumption 
% (e.g. Hansen, 2015), 
(e.g. \citet{hansen2015}), 
but less attention has been devoted to speeding. 
Some empirical research has determined the impact of very influential policies 
like the introduction of a demerit point system. 
For example, 
% Benedettini and Nicita (2009) 
\citet{bennedittiniNicita2009} 
show a reduction in road fatalities 
through deterrence and incapacitation following the introduction of such a system. 
More generally, 
% Castillo-Manzano and Castro-Nuño (2012) 
\citet{castillocastro2012} 
provide a meta-study 
demonstrating the broad positive impact of such a policy in a variety of countries. 
In Quebec, 
% Dionne et al. (2011) 
\citet{dionneetal2011} 
focus on the threat of the loss of license 
on the behaviour of a driver close to the demerit point threshold of suspension. 
They find a reduction in the probability of violation for drivers 
with a large number of demerit points and conclude 
the system is successful in deterring the worst offenders. 
Finally, closest to this paper, 
% Meirambayeva, A. et al. (2014) 
\citet{meirambayeva2014} 
show the introduction of a street-racing law in Ontario decreased the number of accidents 
% by conducting an intervention analysis with an ARIMA model of...
using ARIMA modelling of the monthly number of accidents in Ontario.
% 
% Notice that they model collisions, which is more rare and is aggregated by month. 
% Such an analysis is built on data aggregated across the population, 
% while ours is built on the performance of individual drivers 
% across the population of Quebec at the daily frequency. 
% Although they analyze collisions, we analyze speeding tickets, 
% a less rare event further up the chain of causation
% before accidents and fatalities. 
%
% Also note that they do find the effect for males but not females. 
% By studying an event that happens more frequently---although still rare---we
% can measure the gender differences more precisely. 
% 
% We study an eventy that is earlier in the chain of causation;
% excessive speeding is the cause of many accidents. 
% This event is more common---or, at least, less rare---than collisions, 
% permitting an analysis at the individual level. 


In this paper, we investigate the effect of Quebec’s excessive speeding legislation
 on the frequency and types of violations incurred by Quebec drivers 
using an event-study design. 
Such violations are a proxy for driving behaviour, 
so this study will glean insight into the effect of increasing penalties on dangerous driving. 
It is important to note that we are looking at all violations which result in demerit points, 
and not just those that are affected by the change in the law. 

{\Large Revise roadmap paragraph.}

% Why not begin with data to get the basic facts? 
We begin with a simple theoretical model to examine the predictions of economic theory 
on the effect of the law on drivers. 
% 
The model predicts the possibility of heterogeneous effects by age and gender. 
% 
% Mention data first and model second. 
% 
We then use driving records obtained from administrative data sets 
of the Government of Quebec comprising the universe of violations 
from 2006 to 2010 and records on drivers’ licenses over the same period. 
The use of large administrative datasets is necessary because only a small fraction 
of all drivers are impacted by the policy change; 
yet, these drivers are particularly important because they are 
generally responsible for accidents causing bodily harm and property damage. 
We examine the heterogeneous effects of the excessive speeding law 
on both the extensive margin (getting a ticket) 
and the intensive margin (getting a more severe ticket) across gender and age. 
% 
% What wee find from this analysis is a difference by gender and use it to form
% an economic model that then guides our empirical specification. 

We find that the daily probability of receiving a ticket (extensive margin) 
decreases after the implementation of the law. 
When we examine the results by age group, we find that the effects vary substantially, 
with young drivers between the ages of 16 and 24 being the most affected by the law, 
while there is little effect for drivers over the age of 45. 
Repeating the analysis by gender, we see that both males and females change their behaviour, 
but the magnitude of the effect on males is about eight times that of females. 
Examining the breakdown by age categories, we see that the effect gradually declines 
for males until age 55, while there appears to be no age effect for females. 

We then investigate the effect on the intensive margin. 
For males, the probability of getting tickets worth only one demerit point 
actually increases following the new policy, while tickets for all other point values decrease. 
This result suggests that male drivers are still exceeding the speed limit 
but are driving more slowly than before the introduction of the legislation. 
A similar pattern exists for female drivers for one and two point violations. 
% State the above more precisely. 
We conclude that Quebec’s 2008 excessive speeding law has had substantial spillover effects 
on both the extensive and intensive margins of driving behaviour. 
In other words, not only has it reduced the number of drivers driving well above the speed limit, 
it has also led to a decrease in the propensity to commit other moving violations.

This paper contributes to the literature in several ways. 
It is the first examination using administrative data into the effect of 
an excessive speeding law on driving behaviour 
as proxied by violations by gender and age group. 
Such analysis is important because most countries currently use demerit point systems. 
The question now is not whether these systems work 
but whether and how they can be adjusted to increase road safety. 
Moreover, this paper is to our knowledge the first one to empirically investigate 
the impact of such laws on both the intensive and extensive margins of speeding. 
Finally, by studying the impact of deterrence by gender and age, 
this paper fills a gap acknowledged by 
% Freeman (1999) 
\citet{freeman1999}
on the role of gender in studies surrounding criminality.

{\Large Revise roadmap paragraph.}

The rest of this article is organized as follows. 
Section \ref{sec:Background} covers the details of Quebec’s excessive speeding law 
and the relevant institutional background. 
% We analyze the data and THEN conclude that the model is appropriate for testable predictions. 
A simple theoretical model investigating the effects of the law is constructed in Section \ref{sec:Model}. 
The data and summary statistics are presented in Section \ref{sec:Data}. 
In Section \ref{sec:Empirical}, we conduct the empirical analysis. 
A series of robustness and placebo checks is conducted in Section \ref{sec:Validity}. 
We conclude with a policy discussion in Section \ref{sec:Discussion}.

